\documentclass[11pt]{article}
\usepackage{amsmath}
\usepackage{amsfonts}
\usepackage{amsthm}
\usepackage[utf8]{inputenc}
\usepackage[margin=0.75in]{geometry}

\title{CSC111 Winter 2025 Project 1}
\author{TODO: FILL IN YOUR NAME(S) HERE}
\date{\today}

\begin{document}
\maketitle

\section*{Running the game}
We should be able to run your game by simply running \texttt{adventure.py}.
Ensure all the necessary files are in the same directory:
\begin{itemize}
    \item \texttt{adventure.py}
    \item \texttt{game_entities.py}
    \item \texttt{proj1_event_logger.py}
    \item \texttt{game_data.json}
\end{itemize}

To start the game, execute the following command in the command line:
\begin{lstlisting}
python adventure.py
\end{lstlisting}

Follow the game prompts to enter commands, such as:
\begin{itemize}
    \item Moving in a direction
    \item Picking up items
    \item Interacting with the game environment
\end{itemize}

Enjoy the game!

\section*{Game Map}
Example game map below (edit it to show your actual game map):

\begin{verbatim}
-1 12 -1 -1 -1 -1
7   8  9       11
4  -1  1  9 -1 10
-1 -1  2  -1 -1
-1 -1  3  -1 -1
-1 -1  5  6  -1
\end{verbatim}

Starting location is: 1

\section*{Game solution}
List of commands:
\begin{itemize}
    \item go north, B
    \item take library\_card
    \item go north
    \item go east, C
    \item take laptop\_charger
    \item go north
    \item go west
    \item go south
    \item take notebook
    \item go east
    \item go north
    \item go north
    \item take lab\_key
    \item go south
    \item go south
    \item take USB\_drive
    \item go north
    \item go west
    \item go south
    \item take lucky\_mug
    \item use lucky\_mug
    \item use USB\_drive
    \item use laptop\_charger
\end{itemize}

\section*{Lose condition(s)}
Description of how to lose the game:
In the dataclass Gamestate of adventure.py file, "currstep" variable greater or equal than "maxstep" varible

List of commands:
[1, 2, 2, 3, 5, 5, 6, 10, 11, 11, 9, 8, 7, 7, 8, 8, 12, 12, 8, 9, 9, 1, 9, 9, 9, 8, 12, 8, 9]

Which parts of your code are involved in this functionality:
In adventure.py file, when the player move 1 step(action method), the method "check\_lose" will be run. Whenever  "check\_lose" be run,  "currstep" will be added 1 and there is a if statement to check if "currstep" bigger or equal to maxstep. If yes, '\_display' will be run and print "Sorry, you lose the game! :<", "ongoing" become False and the game end.

\section*{Inventory}

\begin{enumerate}
\item All location IDs that involve items in the game:
[9,5,7,11,2,12,4]
\item Item data:
\begin{enumerate}
    \item For Item 1:
    \begin{itemize}
    \item Item name: USB\_drive
    \item Item start location ID: 9
    \item Item target location ID: 4
    \end{itemize}
        \item For Item 2:
    \begin{itemize}
    \item Item name: laptop\_charger
    \item Item start location ID: 5
    \item Item target location ID: 4
    \end{itemize}
        \item For Item 3:
    \begin{itemize}
    \item Item name: coffee
    \item Item start location ID: 7
    \item Item target location ID: 8
    \end{itemize}
        \item For Item 4:
    \begin{itemize}
    \item Item name: notebook
    \item Item start location ID: 11
    \item Item target location ID: 5
    \end{itemize}
        \item For Item 5:
    \begin{itemize}
    \item Item name: library\_card
    \item Item start location ID: 2
    \item Item target location ID: 5
    \end{itemize}
            \item For Item 5:
    \begin{itemize}
    \item Item name: lab\_key
    \item Item start location ID: 12
    \item Item target location ID: 9
    \end{itemize}
            \item For Item 5:
    \begin{itemize}
    \item Item name: lucky\_mug
    \item Item start location ID: 4
    \item Item target location ID: 4
    \end{itemize}
\end{enumerate}

    \item Exact command(s) that should be used to pick up an item (choose any one item for this example), and the command(s) used to use/drop the item (can copy the list you assigned to \texttt{inventory\_demo} in the \texttt{project1\_simulation.py} file)
Take item: ["go north,B", "inventory", "take library\_card", "inventory"]
Use item: ["go west", "go north", "go west", "go south", "score", "take lucky\_mug", "use lucky\_mug", "score"]

    \item Which parts of your code (file, class, function/method) are involved in handling the \texttt{inventory} command:
\end{enumerate}
Class Gamestate defines inventory variable as list[Item]. Class AdventureGame initializes self.menu which includes "inventory". "\_handle\_menu\_command" checks if the command is "inventory", if yes, run "self.menu\_inventory()". In "menu\_inventory()", it uses list comprehention to get a inventory list from "self.state.inventory" and turn them become string and use "\_display" method to print.

\section*{Score}
\begin{enumerate}

    \item Briefly describe the way players can earn scores in your game. Include the first location in which they can increase their score, and the exact list of command(s) leading up to the score increase:
Players can earn scores by using USB\_drive, laptop\_charger, coffee, notebook, lucky\_mug. It should be noted that if you use the notebook, you will achieve an instant win—this is an Easter egg.

    \item Copy the list you assigned to \texttt{scores\_demo} in the \texttt{project1\_simulation.py} file into this section of the report:
["go west", "go north", "go west", "go south", "score", "take lucky\_mug", "use lucky\_mug", "score"]

    \item Which parts of your code (file, class, function/method) are involved in handling the \texttt{score} functionality:

\end{enumerate}
In class Gamestate, we initialize "score" as int and it default value is 0. Also, "self.menu" of class AdventureGame includes a string called score. Then, if player use "score" command to check their scores, "action" method will be executed and "self.\_handle\_menu\_command" will also be run. After that, "self.menu\_score()" will be run and "self.\_display" will show the score if output = True.


\section*{Enhancements}
\begin{enumerate}
    \item Describe your enhancement \#1 here
    \begin{itemize}
        \item Brief description of what the enhancement is (if it's a puzzle, also describe what steps the player must take to solve it):
        \item Complexity level (choose from low/medium/high):
        \item Reasons you believe this is the complexity level (e.g., mention implementation details, how much code did you have to add/change from the baseline, what challenges did you face, etc.)
    \end{itemize}

    % Uncomment below section if you have more enhancements; copy-paste as needed
    %\item Describe your enhancement here
    %\begin{itemize}
    %    \item Basic description of what the enhancement is:
    %    \item Complexity level (low/medium/high):
    %    \item Reasons you believe this is the complexity level (e.g., mention implementation details)
    %\end{itemize}
\end{enumerate}


\end{document}
